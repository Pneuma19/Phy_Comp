\section{Conclusion}
MOOG is now able to read the formated data. Once when the data is ran through MOOG, the user can determine known abundances at certain wavelengths.
From there you can make comparisons with our Sun, and other stars that we know their elemental abundances, and determine the ratio of heavy elements found in the star. As well, the data can be used to calculate the age of the star. For future work, I would like to include a Blackbody function, also known as a Planck function, and flatten the data. Flattening the data would account for varying sensitivity in the sensors used by HST (and other telescopes).  
